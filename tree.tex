\section{Tree}


\begin{frame}[fragile]{树和二叉树}
  树型结构是结点之间有分支,并且具有层次关系的结构,类似于自然界中的树。树有很多应
  用,比如Unix等操作系统中的目录结构。
\end{frame}

\begin{frame}[fragile]
  \frametitle{例子}
\begin{forest}
 [CEO, for tree={rectangle, minimum width=2cm}, fill=red!10
    [CFO [财务人员] ]
    [CTO [工程师] ]
    [CMO [销售人员] ]
 ]
 \node at (current bounding box.south)
 [below=1ex,draw,cloud,aspect=6,cloud puffs=30]
 {\emph{Simple Company Hierarchy}};
\end{forest}
\end{frame}

\begin{frame}[fragile, plain]
  \scalebox{0.7}{
    \begin{forest}
      [学院, for tree={draw=none, rectangle, minimum width=1cm}, fill=red!10, circle
       [社会学部, grow=west [信息资源管理学院, fill=red!10] [新闻学院] [农业与农村发展学院] [社会与人口学院]
       [公共管理学院] [教育学院]]
       [$\cdots$]
       [ 人文学部,grow=east [哲学院] [文学院] [历史学院] [国学院] [艺术学院] [外国语学院] [清史研究所]]
       ]
       \node at (current bounding box.south)
       [below=1ex,draw,cloud,aspect=6,cloud puffs=30]
       {\emph{人民大学学院设置}};
    \end{forest}
  }
\end{frame}

\begin{frame}[fragile]{内容}
  \begin{easylist} \easyitem
    & 树的基本术语
    & 二叉树
    & 遍历二叉树与线索二叉树
    & 树和森林
    & 哈夫曼树
  \end{easylist}
\end{frame}

\subsection{基本术语}

\begin{frame}[fragile]
  \frametitle{树(TREE)}树(Tree)是$n(n \geq 0)$个结点的有限集$T$。 $T$为空时称为空
  树。当$n>0$时,树有且仅有一个特定的称为根(Root)的结点,其余结点可分为$m(m \geq
  0)$个互不相交的子集$T_1, T_2, \cdots, T_m$,其中每个子集又是一棵树,称为子
  树(Subtree)。
  \begin{enumerate}
  \item 各子树是互不相交的集合。
  \item 除根结点,其它结点有唯一前驱。
  \item   一个结点可以有零个或多个后继。
  \end{enumerate}

  \begin{forest}
    [R, for tree={color=white,fill=black}, fill=red!85
    [A [C] [D] [E]]
    [B [F]]
    ]
  \end{forest}
\end{frame}

\begin{frame}[fragile]
  \frametitle{判断哪些是树结构}
  \includegraphics[width=0.3\textwidth]{dot/tree-judge1.pdf} ~~~~~
  \pause
  \includegraphics[width=0.4\textwidth]{dot/tree-judge2.pdf}
\end{frame}

\begin{frame}[fragile]
  \frametitle{判断哪些是树结构}
  \includegraphics[width=0.35\textwidth]{dot/tree-judge3.pdf} ~~~~~
  \pause
  \includegraphics[width=0.4\textwidth]{dot/tree-judge4.pdf}
\end{frame}

\begin{frame}[fragile]
  \frametitle{树的表示形式}
  \includegraphics[width=0.36\textwidth]{dot/tree-represent1.pdf}  \pause
  \scalebox{0.75}{
    \begin{tikzpicture}[b/.style={fill=black!50},n/.style={minimum width=1cm}]
      \draw node[n] (a) {A} node[b, right=0 of a, minimum width=5cm, fill=red!50]{};	

      \draw node[minimum width=0.5cm, below=0.1 of a](bh){} 
      node[n, right=0 of bh] (b) {B} node[b,fill=blue!50, minimum width=4.3cm,right=0 of b]{};	

      \draw node[minimum width=2cm, below=0.2 of bh](dh){} 
      node[n, right=0 of dh] (d) {D} node[b, minimum width=3.5cm,right=0 of d]{};	

      \draw node[minimum width=3.5cm, below=0.2 of dh](ih){} 
      node[n, right=0 of ih] (i) {I} node[b, fill=green!50, minimum width=2.8cm,right=0 of i]{};	

      \draw node[minimum width=3.5cm, below=0.2 of ih](jh){} 
      node[n, right=0 of jh] (j) {J} node[b, fill=green!50, minimum width=2.8cm,right=0 of j]{};	

      \draw node[minimum width=2cm, below=1.2 of dh](eh){} 
      node[n, right=0 of eh] (e) {E} node[b, minimum width=3.5cm,right=0 of e]{};	

      \draw node[minimum width=2cm, below=1.8 of dh](fh){} 
      node[n, right=0 of fh] (f) {F} node[b, minimum width=3.5cm,right=0 of f]{};		


      \draw node[minimum width=0.5cm, below=3.2 of a](ch){} 
      node[n, right=0 of ch] (c) {C} node[b, fill=blue!50, minimum width=4.3cm,right=0 of c]{};	

      \draw node[minimum width=2cm, below=2.8 of dh](gh){} 
      node[n, right=0 of gh] (g) {G} node[b, minimum width=3.5cm,right=0 of g]{};	
      \draw node[minimum width=2cm, below=3.2 of dh](hh){} 
      node[n, right=0 of hh] (h) {H} node[b, minimum width=3.5cm,right=0 of h]{};	

      \draw node[below=0.1 of h] {凹入表表示法};
    \end{tikzpicture}
      } \pause
  \begin{columns}[t]
    \begin{column}{0.4\textwidth}
      \centering
      \vspace{0pt}
      (A(B(D(I,J),E, F),C(G,H)))
      
      广义表表示 \\
      
      \pause
    \end{column}
    \begin{column}{0.5\textwidth}
\scalebox{0.65}{
    \begin{tikzpicture}[n/.style={ellipse,draw}]
      \draw node[n, minimum width=7.8cm, minimum height=3.5cm, fill=red!5]{}
      node[n, minimum width=4.5cm, minimum height=2.5cm, xshift=-1.2cm, fill=blue!5]{}
      node[n, minimum width=2.5cm, minimum height=1.8cm, xshift=2.5cm, fill=blue!5]{}
      node[n, minimum width=2cm, minimum height=1.8cm, xshift=-2cm, fill=yellow!5]{}
      node[n, circle,xshift=-2.5cm, fill=green!5]{I}
      node[n, circle,xshift=-1.7cm, fill=green!5]{J}
      node[n, circle,xshift=-0.5cm, fill=yellow!5]{E}
      node[n, circle,xshift=0.4cm, fill=yellow!5]{F}
      node[n, circle,xshift=2cm, fill=yellow!5]{G}
      node[n, circle,xshift=3cm, fill=yellow!5]{H}
      node[yshift=1.35cm]{A}
      node[xshift=-0.8cm,yshift=0.8cm]{B}
      node[xshift=2.5cm, yshift=0.6cm]{C}
      node[xshift=-2cm,yshift=0.6cm]{D};
      \draw node[yshift=-2.2cm] {嵌套集合表示};
    \end{tikzpicture}
  }
  
    \end{column}
  \end{columns}
\end{frame}

\begin{frame}[fragile]
  \frametitle{基本术语}
  \begin{columns}[t]
    \begin{column}{0.5\textwidth}
      \begin{itemize}
      \item 树(tree)
      \item 子树(sub-tree)
      \item 结点(node)
      \item 结点的度(degree)
      \item 叶子(leaf)
      \item 孩子(child)
      \item 父亲(parents)
      \item 兄弟(sibling)
      \item 祖先
      \item 子孙
      \end{itemize}
    \end{column}
    \begin{column}{0.5\textwidth}
      \begin{itemize}
      \item 树的度(degree)
      \item 结点的层次(level)
      \item 树的深度(depth)
      \item 有序树
      \item 无序树
      \item 森林
      \end{itemize}

      \begin{forest}
        [R
        [A, 
        [C] [D]]
        [B [E]]
        ]
      \end{forest}
    \end{column}
  \end{columns}
\end{frame}

\begin{frame}[fragile, plain]
~
\end{frame}

\subsection{二叉树}
\begin{frame}[fragile]
  \frametitle{二叉树(Binary Tree)}二叉树的一个重要应用是在查找中的应用。当然,它
  还  有许多与搜索无关的重要应用,比如在编译器的设计领域。

  \begin{itemize}
  \item 二叉树是一种树型结构,它的每个结点至多只有两个子树,分别称为左子树和右子树。
    二叉树是有序树。
  \item 二叉树是$n(n \geq 0)$个结点构成的有限集合。二叉树为空,或是由一个根结点及
    两棵互不相交的左右子树组成,并且左右子树都是二叉树。
  \item 在二叉树中要区分左子树和右子树,即使只有一棵子树。这是二叉树与树的最主要的
    差别。
  \end{itemize}
\end{frame}

\begin{frame}[fragile]
  \frametitle{二叉树的五种形态}
  \begin{enumerate}
  \item 空二叉树;
  \item 只有根结点(左右子树都为空);
  \item 只有左子树(右子树为空);
  \item 只有右子树(左子树为空);
  \item 左右子树均不空。
  \end{enumerate}

  \scalebox{0.8}{
    \begin{tikzpicture}[n/.style={draw=black!80, thick,fill=black!80, circle, minimum size=1cm},
      t/.style={draw=black!80, ellipse, minimum height=2cm,minimum width=1cm, fill=black!20}]
      \draw node[n,dotted, fill=yellow!1] (t1) at (0,0) {};
      \draw[draw] (0.7,0.7) -- (-0.7,-0.7);

      \draw node[n] (t2) at (2,0) {};

      \draw node[n] (t3) at (4.5,0) {} node[t, below left=of t3,xshift=0.8cm, rotate=-30] (t31) {};
      \draw[draw] (t3) -- (t31);

      \draw node[n] (t4) at (6.5,0) {} node[t, below right=of t4,xshift=-0.8cm, rotate=30] (t41) {};
      \draw[draw] (t4) -- (t41);

      \draw node[n] (t5) at (11,0) {} node[t, below left=of t5,xshift=0.8cm, rotate=-30] (t51) {} node[t, below right=of t5,xshift=-0.8cm, rotate=30] (t52) {};
      \draw[draw] (t5) -- (t51);
      \draw[draw] (t5) -- (t52);
    \end{tikzpicture}
  }
\end{frame}

\begin{frame}[fragile]
  \frametitle{请观察二叉树, 并回答下列问题}

  \scalebox{0.7} {
    \begin{forest}
      [A, name=A
      [B, name=B [D [H] [I]] [E [J] [K]]]
      [C, name=C [F [L] [M]] [G [N] [O]]]
      ]
     \draw[draw, dotted, thick] (B.north) ++(-2cm, 0.2cm)-- ++(9cm, 0cm)
     node[above, right, yshift=0.5cm]{1};
     \draw[draw, dotted, thick] (B.south) ++(-2cm, -0.2cm)-- ++(9cm, 0cm)
     node[above, right, yshift=0.5cm]{2};
     \draw[draw, dotted, thick] (B.south) ++(-2cm, -1.6cm)-- ++(9cm, 0cm)
     node[above, right, yshift=0.5cm]{3} node[below, right, yshift=-0.5cm] {4};
    \end{forest}
  }

  \begin{enumerate}
  \item 二叉树的第$i$层最多有多少个结点?
  \item 二叉树深度为k,则它最多有多少个结点?
  \item 二叉树有n个节点,请问它最小深度是几?
  \item 二叉树叶子的数目和度为2的节点的数目是否相等?如果不等,又是什么关系?
  \end{enumerate}
\end{frame}

\begin{frame}[fragile]
  \frametitle{二叉树的性质}
  \begin{itemize}
  \item 性质1: 二叉树的第$i$层至多有$2^{i-1}$个结点。
  \item 性质2: 深度为$k$的二叉树至多有$k2^{k-1}$个结点($k \geq 1$)。
  \item \color{red} 性质3: 二叉树中终端结点数为$n_0$,度为2的结点数为$n_2$,则有$n_0 = n_2 + 1$
    (试证明)
  \end{itemize}
\end{frame}

\begin{frame}[fragile]
  \frametitle{二叉树的性质}
  二叉树中终端结点数为$n_0$,度为2的结点数为$n_2$,则有$n_0 = n_2 + 1$

  \begin{columns}[T,c]
    \begin{column}{0.6\linewidth}
      \begin{itemize}
      \item 设二叉树中度为1的结点数为$n_1$, 二叉树中总结点数为$N$,则有:
        
        $N = n_0 + n_1 + n_2$
      \item 再考虑二叉树中的分支数(每个节点有唯一一个入的分支,根节点除外;再考虑出
        的分支数量), 则有:

        $N - 1 =n_1 + 2 \times n_2$
      \item 整理可得:

        $n_0 = n_2 + 1$
      \end{itemize}
    \end{column}
    \begin{column}{0.35\linewidth}      
      \begin{forest}
        [{}, name=root,for tree={color=white,fill=black}
        [{}, name=n21, grow=-55 [{}, name=n31 [{~}] [{~}]]]
        [{}, name=n22, grow=235 [{}, name=n32,[{~}, draw=none,fill=none, no edge] [{~}]]]
        ]
      \end{forest}
    \end{column}
  \end{columns}
\end{frame}

\begin{frame}[fragile]{}
\begin{forest}
  for tree={draw, circle, l sep+=0.5em, where level={3}{s sep+=1em}{}, inner sep=0.05em}
  [,fill
      [{$-$}, edge label={node[midway,left]{$0.70$}}, name=e
          [$-$, edge label={node[midway,left]{$0.78$}}, name=a
              [$-$, edge label={node[midway,left]{$0.80$}}, name=c
                  [$-$, edge label={node[midway,left]{$0.82$}}, name=d]
                  [$+$, edge label={node[midway,right]{$0.18$}}]
              ]
              [$+$, edge label={node[midway,right]{$0.20$}}
                  [$-$, edge label={node[midway,left]{$0.49$}}]
                  [$+$, edge label={node[midway,right]{$0.51$}}]
              ]
          ]
          [$+$, edge label={node[midway,right]{$0.22$}}
              [$-$, edge label={node[midway,left]{$0.50$}}
                  [$-$, edge label={node[midway,left]{$0.70$}}]
                  [$+$, edge label={node[midway,right]{$0.30$}}]
              ]
              [$+$, edge label={node[midway,right]{$0.50$}}
                  [$-$, edge label={node[midway,left]{$0.57$}}]
                  [$+$, edge label={node[midway,right]{$0.43$}}]
              ]
          ]
      ]
      [{$+$}, edge label={node[midway,right]{$0.30$}}
          [$-$, edge label={node[midway,left]{$0.50$}}
              [$-$, edge label={node[midway,left]{$0.71$}}
                  [$-$, edge label={node[midway,left]{$0.75$}}]
                  [$+$, edge label={node[midway,right]{$0.25$}}]
              ]
              [$+$, edge label={node[midway,right]{$0.29$}}
                  [$-$, edge label={node[midway,left]{$0.53$}}]
                  [$+$, edge label={node[midway,right]{$0.47$}}]
              ]
          ]
          [$+$, edge label={node[midway,right]{$0.50$}}
              [$-$, edge label={node[midway,left]{$0.50$}}
                  [$-$, edge label={node[midway,left]{$0.72$}}]
                  [$+$, edge label={node[midway,right]{$0.28$}}]
              ]
              [$+$, edge label={node[midway,right]{$0.50$}}, name=b
                  [$-$, edge label={node[midway,left]{$0.44$}}]
                  [$+$, edge label={node[midway,right]{$0.56$}}]
              ]
          ]
      ]
  ]
  \begin{scope}[on background layer]
    \node [fill=black!25, fit={(a.south) (c) (b)}, inner ysep=.05em, inner xsep=.75em, outer sep=0pt] {};
  \end{scope}
  \node (p) [left=5pt of d] {memory 4};
  \node at (p |- c) {memory 3};
  \node at (p |- a) {memory 2};
  \node at (p |- e) {memory 1};
\end{forest}


  \begin{easylist} \easyitem

\end{easylist}
\end{frame}


